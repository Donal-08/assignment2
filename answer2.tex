\documentclass[journal,12pt,twocolumn]{IEEEtran}

\usepackage{enumitem}
\usepackage{amsmath}
\usepackage{amssymb}
\usepackage{graphicx}
\providecommand{\sbrak}[1]{\ensuremath{{}\left[#1\right]}}

\title{Assignment 2 \\ \Large AI1110: Probability and Random Variables \\ \large Indian Institute of Technology Hyderabad}
\author{Donal Loitam \\ \normalsize AI21BTECH11009 \\ \vspace*{20pt} \normalsize  12 April 2022 \\ \vspace*{20pt} \Large ICSE 2019 Grade 12}


\begin{document}
	% The title
	\maketitle
	
	% The question
	\textbf{Question 1(vi)} 
	Prove that the function $f(x)=x^3-6x^2+12x+5$ is increasing on $\mathbb{R}$
	
	% The solution
	\textbf{Solution.}
	Given function is:
  \begin{align}
      f(x)=x^3-6x^2+12x+5
  \end{align}
	
	Using the first principle of differentiation,the first derivative of $f(x)$:
  \begin{align}
	 f'(x)&=\lim_{h \to 0} \tfrac{f(x+h)-f(x)}{h} \\
    \begin{split}
           &=\lim_{h \to 0} \left[\tfrac{(x+h)^3 - 6(x+h)^2+12(x+h)-x^3}{h}\right.\\
	   &\hspace{8em}\left.+\tfrac{6x^2-12x}{h}\right]\\
     \end{split}\\
     \begin{split}
          &=\lim_{h \to 0} \tfrac{\sbrak{(x+h)^3-x^3}-6\sbrak{(x+h)^2-x^2}}{h}\\
	  & \hspace{8em}+\tfrac{12\sbrak{(x+h)-x}}{h} \\
      \end{split}\\
          \because(x+h)^3&=x^3+3x^2h+3xh^2+h^3\\
           f'(x)&=\lim_{h \to 0} \tfrac{\sbrak{h^3+3x^2h+3xh^2}-6\sbrak{2xh+h^2}+12h}{h} \\
                &=\lim_{h \to 0} \tfrac{h^3+3x^2h+3xh^2-12xh-6h^2+12h}{h} \\
       \begin{split}
	      &= \lim_{h \to 0} (h^2 + 3x^2 + 3xh - 12x \\
	      & \hspace{8em} - 6h + 12) \\
	\end{split} \\
    (\because{h \to 0}&,{h^2 \to 0})\\
             f'(x) &=\lim_{h \to 0} (3x^2-12x+12) \\ 
     \implies f'(x)&=3x^2-12x+12\\ 
     \implies f'(x)&=3(x^2-4x+4) \\
     \implies f'(x)&=3(x-2)^2 \\
    \therefore\;\;\;  f'(x)&\ge0 \;\;\;,\;\forall ~x \in \mathbb{R}
  \end{align}
	
  Since the slope of $f(x)$ i.e $f'(x)\ge0, \;\;\forall ~x \in \mathbb{R}$      \\ 
	$\therefore$\; $f(x)=x^3-6x^2+12x+5$ is always increasing on $\mathbb{R}$
	
	
	% The graph
	\begin{figure}[!ht]
		\centering
		\includegraphics[width=\columnwidth]{mygraph.png}
		\caption{Graph of $f(x) = x^3-6x^2+12x+5$ }
		\label{fig-1}
	\end{figure}
	
\end{document}
